\chapter{Integrating clinical and biological data to predict treatment response in individuals with anorexia nervosa.}\label{ch:singleCellSeq}

\section[Abstract]{Abstract\footnotemark}

Maintenance and recovery after hospitalization for anorexia nervosa (AN) is challenging, and up to 45\% of patients are rehospitalized. Better predictions of outcome post-discharge-- specifically negative outcomes, which are less studied than who does well-- could enable earlier and more personalized interventions. To develop predictive models, we obtained clinical, demographic, and biological information from 29 females with AN over 15 years-old at admission, discharge, and 3 timepoints post-discharge from the UNC Eating Disorder Unit. We could not roboustly predict BMI, change in BMI, or rehospitalization at any time point. Prediction in this space will

Our findings around clinical outcome, however, point to new areas for research. Consistent with previous findings, BMI at discharge is well correlated with BMI post-discharge (Pearson?s r = 0.60, 0.53, 0.41 for 3-, 7- and 12-months post-discharge, respectively, P \textless 0.01 for each). Surprisingly, we found that BMI at discharge is lowly correlated with rate of BMI change (r = 0.19, 0.10, -0.05, respectively, P \textgreater 0.29 for each), which was consistent in an independent cohort. These results suggest that change in BMI may play an important role in rehospitalization. Based on our results, we suggest a framework for predicting treatment response in individuals with anorexia nervosa.

\footnotetext{Citation for chapter: Lauren E Blake, Rachel Guerra, Christopher Hubel, Laura M. Thorton, Tae Kim, Yuxin Zou, Patrick F. Sullivan, Cynthia M. Bulik, Jessica H. Baker. Integrating clinical and biological data to predict treatment response in individuals with anorexia nervosa. \textit{Manuscript in prep.}}



\subsection{Author contributions}

JHB conceived of the study and designed the experiments, with input from PFS and CMB. RG and JHB collected the data. LEB analyzed the data, with input from CH, LMT, PFS, TK, YZ, and CMB. LEB and JHB wrote the paper. 

