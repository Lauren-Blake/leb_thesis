\chapter{Integrating clinical and biological data to predict treatment response in individuals with anorexia nervosa.}\label{ch:singleCellSeq}

\section[Abstract]{Abstract\footnotemark}

Maintenance and recovery after hospitalization for anorexia nervosa (AN) is challenging, and up to 45\% of patients are rehospitalized. Better predictions of outcome post-discharge-- specifically negative outcomes, which are less studied than who does well-- could enable earlier and more personalized interventions. To develop predictive models, we obtained clinical, demographic, and biological information from 29 females with AN over 15 years-old at admission, discharge, and 3 timepoints post-discharge from the UNC Eating Disorder Unit. We could not roboustly predict BMI, change in BMI, or rehospitalization at any time point. Prediction in this space will

Our findings around clinical outcome, however, point to new areas for research. Consistent with previous findings, BMI at discharge is well correlated with BMI post-discharge (Pearson?s r = 0.60, 0.53, 0.41 for 3-, 7- and 12-months post-discharge, respectively, P \textless 0.01 for each). Surprisingly, we found that BMI at discharge is lowly correlated with rate of BMI change (r = 0.19, 0.10, -0.05, respectively, P \textgreater 0.29 for each), which was consistent in an independent cohort. These results suggest that change in BMI may play an important role in rehospitalization. Based on our results, we suggest a framework for predicting treatment response in individuals with anorexia nervosa.

\footnotetext{Citation for chapter: Lauren E Blake, Rachel Guerra, Christopher Hubel, Laura M. Thorton, Tae Kim, Yuxin Zou, Patrick F. Sullivan, Cynthia M. Bulik, Jessica H. Baker. Integrating clinical and biological data to predict treatment response in individuals with anorexia nervosa. \textit{Manuscript in prep.}}

\section{Introduction}\label{ch04-introduction}

\subsection{Predicting clinical outcome in anorexia nervosa (AN)}

\subsection{Integrating biological factors in predictive models of clinical outcome}

\subsection{Gene expression levels in individuals with AN}

\section{Results}\label{ch04-results}

\subsection{Study design and sample characteristics}\label{study-design-and-quality-control}

Description of study design, collected biological, clinical/demographic, and technical factors (see more information in Methods, Supplement)
Low number of individuals in the focus sample (at admission and discharge)
Unfortunately, attrition is low for later points (which is important when designing future studies)

\subsection{Gene expression levels during hospitalization}\label{batch-effects-associated-with-umi-based-single-cell-data}


Gene expression levels collection and processing information, technical factor analysis
Lots of DE genes during hospitalization (T1 versus T2) but few remain when we include weight change as a covariate (see supplement) ? suggests strong genomic response to nutritional rehabilitation
Question of utility/interpretation of these DE genes, given tissue type and confounding of days on the unit and weight gain. 
Comment on using rehospitalization status (for DE or for predictive modeling)- need to look at how many individuals ended up back in the hospital.
We then debated identifying genes that were differentially expressed between individuals that underwent ?weight relapse? and those that did not
No consensus definition for weight relapse: cannot examine group differences
It was difficult to determine what was consequential weight loss e.g. weight loss of 0 pounds, 3 pounds, 5 pounds, etc. 

\subsection{Identification of target variables}\label{measuring-regulatory-noise-in-single-cell-gene-expression-data}

Then started to look at weight patterns- first looked at trajectories
Trajectories had high intraindividual ranges and were inconsistent (e.g. very few individuals that consistently gained or lost during the 1 year post-discharge)
Then looked at moments of the BMI distribution from discharge to 12 months post-discharge
Used BMI (standardized weight) instead of raw weight
Average weight (discharge to 12 months post-discharge) hard to interpret clinically
Chose BMI at 3-, 7-, and 12-months post-discharge and change in BMI/month as target variables

\subsection{Failure to predict target variables}\label{prediction-target-variables}


\section{Discussion}\label{ch04-discussion}

\subsection{Lessons learned}\label{study-design-and-sample-size-for-scrna-seq}

\subsection{Big data in eating disorders}\label{the-limitations-of-the-ercc-spike-in-controls}


\section{Methods}\label{ch04-methods}

\subsection{Ethics statement}\label{ch04-ethics-statement}


\subsection{Sample collection}\label{cell-culture-of-ipscs}


\subsection{Blood processing and RNA-sequencing}\label{single-cell-capture-and-library-preparation}


\subsection{Obtaining gene expression levels and differential expressed (DE) genes}\label{illumina-high-throughput-sequencing}


\subsection{Methods for prediction}\label{read-mapping}



\subsection{Data and code
availability}\label{ch04-data-and-code-availability}

When the paper is available as a preprint, the data will be deposited in NCBI's Gene Expression Omnibus and the code and processed data will be available on Lauren Blake's github account. 

\section{Acknowledgments}\label{ch04-acknowledgments}

We thank members of the Pritchard, Gilad, and Stephens laboratories
for valuable discussions during the preparation of this
manuscript. This work was funded by NIH grant HL092206 to YG and HHMI
funds to JKP. PYT is supported by NIH T32HL007381. JDB was supported
by NIH T32GM007197.  The content is solely the responsibility of the
authors and does not necessarily represent the official views of the
funding bodies. 

