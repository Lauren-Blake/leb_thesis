\chapter{Integrating clinical and biological data to predict treatment response in individuals with anorexia nervosa.}\label{ch:singleCellSeq}

\section[Abstract]{Abstract\footnotemark}

Single-cell RNA sequencing (scRNA-seq) can be used to characterize
variation in gene expression levels at high resolution. However, the
sources of experimental noise in scRNA-seq are not yet well
understood.  We investigated the technical variation associated with
sample processing using the single-cell Fluidigm C1 platform. To do
so, we processed three C1 replicates from three human induced
pluripotent stem cell (iPSC) lines. We added unique molecular
identifiers (UMIs) to all samples, to account for amplification
bias. We found that the major source of variation in the gene
expression data was driven by genotype, but we also observed
substantial variation between the technical replicates. We observed
that the conversion of reads to molecules using the UMIs was impacted
by both biological and technical variation, indicating that UMI counts
are not an unbiased estimator of gene expression levels. Based on our
results, we suggest a framework for effective scRNA-seq studies.

\footnotetext{Citation for chapter: Lauren E Blake, Rachel Guerra, Christopher Huebel, Laura M. Thorton, Tae Kim, Yuxin Zou, Patrick F. Sullivan, Cynthia M. Bulik, Jessica H. Baker. Integrating clinical and biological data to predict treatment response in individuals with anorexia nervosa. \textit{Manuscript in prep.}}

\section{Introduction}\label{ch04-introduction}

\subsection{Predicting clinical trajectory in anorexia nervosa}

\subsection{Previous studies}

\section{Results}\label{ch04-results}

\subsection{Study design and sample characteristics}\label{study-design-and-quality-control}

Description of study design, collected biological, clinical/demographic, and technical factors (see more information in Methods, Supplement)
Low number of individuals in the focus sample (at admission and discharge)
Unfortunately, attrition is low for later points (which is important when designing future studies)

\subsection{Gene expression levels during hospitalization}\label{batch-effects-associated-with-umi-based-single-cell-data}


Gene expression levels collection and processing information, technical factor analysis
Lots of DE genes during hospitalization (T1 versus T2) but few remain when we include weight change as a covariate (see supplement) ? suggests strong genomic response to nutritional rehabilitation
Question of utility/interpretation of these DE genes, given tissue type and confounding of days on the unit and weight gain. 
Comment on using rehospitalization status (for DE or for predictive modeling)- need to look at how many individuals ended up back in the hospital.
We then debated identifying genes that were differentially expressed between individuals that underwent ?weight relapse? and those that did not
No consensus definition for weight relapse: cannot examine group differences
It was difficult to determine what was consequential weight loss e.g. weight loss of 0 pounds, 3 pounds, 5 pounds, etc. 

\subsection{Identification of target variables}\label{measuring-regulatory-noise-in-single-cell-gene-expression-data}

Then started to look at weight patterns- first looked at trajectories
Trajectories had high intraindividual ranges and were inconsistent (e.g. very few individuals that consistently gained or lost during the 1 year post-discharge)
Then looked at moments of the BMI distribution from discharge to 12 months post-discharge
Used BMI (standardized weight) instead of raw weight
Average weight (discharge to 12 months post-discharge) hard to interpret clinically
Chose BMI at 3-, 7-, and 12-months post-discharge and change in BMI/month as target variables

\subsection{Attempting to predict target variables}\label{prediction-target-variables}


\section{Discussion}\label{ch04-discussion}

\subsection{Clinically actionable response variables}\label{study-design-and-sample-size-for-scrna-seq}

\subsection{Outlook: Big data in eating disorders}\label{the-limitations-of-the-ercc-spike-in-controls}


\section{Methods}\label{ch04-methods}

\subsection{Ethics statement}\label{ch04-ethics-statement}

The YRI cell lines were purchased from CCR. The original samples were
collected by the HapMap project between 2001-2005. All of the samples
were collected with extensive community engagement, including
discussions with members of the donor communities about the ethical
and social implications of human genetic variation research. Donors
gave broad consent to future uses of the samples, including their use
for extensive genotyping and sequencing, gene expression and
proteomics studies, and all other types of genetic variation research,
with the data publicly released.

\subsection{Sample collection}\label{cell-culture-of-ipscs}


\subsection{Blood processing and RNA-sequencing}\label{single-cell-capture-and-library-preparation}


\subsection{Obtaining gene expression levels and differential expressed (DE) genes}\label{illumina-high-throughput-sequencing}


\subsection{Methods for prediction}\label{read-mapping}



\subsection{Data and code
availability}\label{ch04-data-and-code-availability}

When the paper is available as a preprint, the data will be deposited in NCBI's Gene Expression Omnibus and the code and processed data will be available on Lauren Blake's github account. 

\section{Acknowledgments}\label{ch04-acknowledgments}

We thank members of the Pritchard, Gilad, and Stephens laboratories
for valuable discussions during the preparation of this
manuscript. This work was funded by NIH grant HL092206 to YG and HHMI
funds to JKP. PYT is supported by NIH T32HL007381. JDB was supported
by NIH T32GM007197.  The content is solely the responsibility of the
authors and does not necessarily represent the official views of the
funding bodies. 

