\chapter{Conclusion}\label{conclusion}

\subsection{Eating disorder genetics}
A major promise of the field of human genetics is to better understand human complex traits and diseases. As described in previous chapters, a properly designed study can facilitate this biological understanding. In Chapter 2, the use of matched tissues between species allowed us to estimate the contribution of DNA methylation to conserved inter-tissue differences in gene expression. We also found that conserved gene regulatory differences are more likely to be involved in gene regulatory networks and protein-protein networks than non-conserved differences. In Chapter 3, we leveraged matched induced Pluripotent Stem Cell (iPSC) panels from humans and chimpanzees. In this developmental timecourse from iPSCs to the endoderm stage, my collaborators and I found high conservation during the differentiation. Furthermore, we reported a reduction in regulatory variation at the primitive streak stage in a significant number of genes. Due to our study design and careful analysis of technical variables, we were able to provide evidence that this canalization is driven by biological, rather than technical, factors. Finally, in Chapter 4, I discussed power to identify biologically-based predictors in the context of a longitudinal study design. 
Study design plays a critical role in all types of comparative studies, whether the goal is to understand differences between tissues, species, or diseased and healthy individuals. In particular, genomic studies can be very useful for understanding complex traits and diseases, including eating disorders. When designed properly, genomic studies can uncover biological mechanisms and may be used for prediction related to the disorder and treatment. For example, understanding the underlying biological mechanisms for a given disorder can help to develop new drug targets \cite{RN40}. This course of action is particularly important for eating disorders, as few medications are available specifically for BN or BED, and no pharmacological interventions have been approved specifically for AN treatment \cite{RN1060}. This line of research could also help the field gain a greater understanding of how existing pharmacological interventions work \cite{RN40}. There may be treatments that could be developed to ease the process of recovery for AN patients, including nutrient absorption and the pain experienced during nutritional rehabilitation \cite{RN1411}.
Another developing area within eating disorders is that of prediction. Determining risk of eating disorders, responses to treatment, and risk for relapse has major implications for screening \cite{RN4543}. Integrating genomics into one?s risk profile could increase prediction performance metrics, which will ultimately increase patient care. Prediction is also important for personalized medicine. Currently, individuals with eating disorders are often treated with therapeutics for comorbid conditions. Unfortunately, these drugs are not tested in very low weight individuals, such as those with AN. Since weight influences metabolism, some clinicians use genetics to inform which therapeutic to prescribe \cite{RN4212}. As additional therapeutics become available for the treatment of eating disorders and common comorbidities, individuals could be prescribed an intervention(s) shown to be most effective for their clinical and genetic profile. 
	Ultimately, our efforts to understand the biology of eating disorders will have a profound social and psychological impact on eating disorder patients and their families. Like many psychiatric disorders, eating disorders are both highly stigmatized and poorly understood \cite{RN4212}. Therefore, diagnoses often lead to feelings of shame and guilt \cite{RN1089}. These feelings can be alleviated by understanding the neurobiological and genetic causes for these disorders, providing considerable and immediate relief to patients and their families \cite{RN4212}.
\\
I will now describe recent progress in eating disorder genomics and consider key challenges in uncovering the biological mechanisms of eating disorders and in predicting eating disorder outcomes. From my discussion of these challenges, I also identify opportunities to improve future genomic studies of eating disorders. 

\subsection{Progress on the use of genetics to understand the biological mechanisms of eating disorders}
Research into the genetic risk factors of eating disorders is still in its early stages \cite{RN4212}. Furthermore, the majority of eating disorder research has focused on AN. The second and most recent genome wide association study (GWAS) of AN found the heritability of AN to be 11-17\% \cite{RN4568}. In this study of over 16,000 AN cases and 55,000 eating disorder controls, 8 SNPs were associated with differences between AN cases and controls \cite{RN4568}. Some of these SNPs associated with AN cases have also been associated with metabolic traits, such as body mass index (BMI) \cite{RN4568}. 

\subsection{Challenges and possible solutions going forward}
While the AN GWAS represents major progress, it also highlights many of the challenges to the field going forward. All 8 significant hits were located in non-protein coding regions \cite{RN4115}. Consequently, it is hard to dissect the impact of these SNPs on disease risk. This interpretation is particularly difficult because these genes may impact multiple tissues, or even have different effects in different tissues. Unfortunately, the functional follow-up for AN is incredibly difficult, as it is difficult to access human brain tissues and there are no widely accepted mouse models of AN \cite{RN1060}.  
To overcome challenges around access to human brain tissues and functional followup, 
researchers in other psychiatric fields have generated iPSC-based models from cases and controls \cite{RN2029, RN2026}. A preliminary study demonstrated that it is possible to derive neurons from iPSCs in AN cases and controls \cite{RN2030}. However, interpretation again is difficult: it hard to know what phenotype to measure in these ?neurons in a dish?, particularly for AN. While this original study focused on gene expression levels, iPSC-derived neurons from individuals with schizophrenia have been assessed for other properties as well \cite{RN2022}. 
	The eating disorder field, and AN in particular, also faces challenges around comorbities. For example, there are well documented phenotypic comorbidities between eating disorders and other psychiatric disorders \cite{RN34}. These phenotypic correlations are also reflected by strong genetic correlations \cite{RN4568}. Furthermore, AN is confounded with low BMI \cite{RN4522}. Since BMI and disease state are confounded, it is difficult to direct assess claims about the metabolic etiology of AN. To attempt to overcome the strong relationship between BMI and AN, \cite{RN4568} compares AN patients to approximately 1,500 constituently thin individuals. While this approach is a good first step, I argue that this approach will be limited in the long term, particularly because there is little transcriptomic, proteomic, and metabolomic data on individuals with AN \cite{RN1411}. To make significant gains on this front, more work needs to be done to understand the multi-faceted contributors to weight more generally.  
While comorbidities complicate the study of eating disorders, the application of newer analytical methods leverage the complexities. For example, to partition variants associated with a psychiatric disorder specifically, compared to more generally, case-case GWAS are currently applied to other psychiatric disorders \cite{RN4574}. Case-case GWAS may be useful to compare across eating disorders. However, a more interesting application could be to apply this technique to individuals with a given sub-phenotype, such as individuals with anxiety but no history of eating disorders compared to individuals with anxiety and eating disorders. The results of this type of study may be helpful for differentiating the multiple components of eating disorders. Overall, understanding the underlying mechanisms of disorders as complex as eating disorders will require substantial long-term investment and multiple channels of inquiry. 

\subsection{Progress on the integration of genetics in predictive models in individuals with eating disorders}
	While understanding biological mechanisms of disease is a critical course of action, prediction represents a more immediately clinically actionable path. While many studies have attempted to predict long term outcome for individuals with eating disorders, to my knowledge, none have integrated non-hormone-based genetic information \cite{RN1168, RN1167, RN4514, RN4543, RN4539}. Instead, genetic information has been used to predict risk of AN, rather than long term outcome, primarily through polygenic risk score (PRS) \cite{RN40}. In the most recent AN GWAS, PRS was applied to explain 1.7\% of phenotypic variation \cite{RN4568}. This low percentage of variance explained suggests a current limited clinical utility. PRS has not been calculated for different subtypes or outcomes of AN, which is likely due to sample size. Even if the PRS is higher for these responses, it will be important to integrate clinical and other phenotypic information into the predictors. 

\subsection{Challenges and possible solutions going forward}
One of my fears for using PRS for disorder risk is the potential for misinterpretation and misuse \cite{RN1127}. Individuals may over interpret an increased risk for a disorder as genetic determinism, which is highly problematic for an already stigmatized class of disorders such as eating disorders \cite{RN1175, RN4209, RN111, RN1137}. Therefore, I argue that it is more beneficial to focus on individuals that have already been diagnosed with an eating disorder. In these individuals, prediction could directly impact treatment options. 
As described in more detail in Chapter 4, most outcome studies in AN do not take into account genetic factors (including hormones). These studies face difficult, but not unsurmountable challenges related to study design, defining outcomes, choice of predictors, and analysis methods. As argued in Chapter 4, the first priority should be to determine more rigorous and clinically relevant definition of outcomes for eating disorders, such as quantifying risk of rehospitalization for individuals with AN. 
Going forward, principles learned from more mature fields could be serve as useful guides for the field of eating disorder genomics. For example, extensive record keeping of both biological and technical variables, such as that discussed in Chapter 2, is just beginning to gain traction in the eating disorders genomics field \cite{RN4569}. Furthermore, measures of cellular heterogeneity were extremely important to account for in the iPSC study from Chapter 3. Controlling for cell type heterogeneity is critical when using whole blood from patients, such as in Chapter 4 and \cite{RN1411}. These two examples highlight the importance of cross-talk between multiple genomics fields. 

\subsection{Final words}
In conclusion, progress in complex disorders such as eating disorders can only be achieved by integrating ?big data? sources, including genomics, into well designed studies. To do so will require a major commitment by federal funding and other major funding agencies to support basic scientists and subsequently, strong collaborations between basic and clinician scientists.

