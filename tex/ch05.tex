\chapter{Conclusion}\label{conclusion}

Traditional genetics approaches have been unable to identify variants
which can be used to predict susceptiblity to tuberculosis (TB),
likely due to the highly polygenic architecture of this complex trait
\citep{Thye2010, Mahasirimongkol2012, Thye2012, Png2012, Chimusa2014,
  Curtis2015, Sobota2016}.  Thus I performed experiments to
interrogate a higher level phenotype, gene expression levels, for
which the effect of many variants of small effect size can manifest in
aggregrate. In my first approach, I identified genes in innate immune
cells whose gene expression levels change in response to infection
with \emph{Mycobacterium tuberculosis} (MTB) but not other bacteria,
highlighting their potential importance for mycobacterial diseases
\citep{Blischak2015}. In my second approach, I measured gene
expression levels in innate immune cells from individuals either
susceptible or resistant to develop active TB and built a classifier
to predict susceptibility to TB. These first two experiments measured
average gene expression levels across many cells, and thus they missed
any cell-to-cell heterogeneity in the innate immune system
\citep{Satija2014, Proserpio2016}. In my third approach, I established
principles for the effective design of studies to measure gene
expression levels in single cells \citep{Tung2016}. Given the success
of my first two experiments, I expect many more discoveries will be
made by interrogating gene expression measurements in single cells of
the innate immune system.

\section{A joint Bayesian model provides a general framework for analyzing functional genomics studies with many conditions}

In Chapter \ref{ch:tb}, I described my work investigating the innate
immune response to MTB \citep{Blischak2015}. It is known that the
innate immune response is important for fighting MTB infections
\citep{Khan2016}. Alveolar macrophages are the primary target of MTB,
and they initiate the formation of granulomas to sequester MTB
\citep{Sia2015}. Furthermore, vaccines against TB have had limited
efficacy \citep{Wang2013}. To identify human genes which are important
for the response to MTB infection, we isolated macrophages from six
healthy donors, infected them with MTB and other bacteria, and
measured genome-wide gene expression levels using RNA-seq at 4, 18,
and 48 hours post-infection.

Previous studies had identified genes which are differentially
expressed upon infection with MTB \citep{Ehrt2001, Ragno2001, Nau2002,
  Chaussabel2003, Volpe2006, Tailleux2008}, and some have even
compared the differences between the reponse to strains of MTB that
vary in their virulence \citep{Coscolla2010, Wu2012}. The first
novelty of our study was to include many other bacteria in the
infection experiments. Specifically, we included the following
mycobacteria: two strains of virulent MTB, avirulent
(heat-inavtivated) MTB, bacillus Calmette-Gu\'{e}rin (BCG; attentuated
\emph{Mycobacterium bovis} used as a vaccine), and the avirulent
\emph{Mycobacterium smegmatis}. The non-mycobacteria species we
included were \emph{Yersinia pseudotuberculosis}, \emph{Salmonella
  typhimurium}, and \emph{Staphylococcus epidermidis}. This allowed us
to distinguish between the innate immune response to MTB versus other
virulent bacteria, MTB versus avirulent mycobacteria, amd MTB versus
deceased MTB.

This novel study design comparing many bacterial infections to isolate
the innate immune respone to MTB also posed analytical challenges. The
goal was to identify differences between the innate immune response to
each of the eight bacterial infections compared to the non-infected
control condition.  Standard differential expression analyses (or in
general any large scale testing of thousands or more genomic features)
are well-suited for experiments with a few conditions
\citep{Oshlack2010, Anders2013, Ritchie2015}. For example, the most
common approach is to perform pairwise differential expression tests
and then overlap the lists of differentially expressed genes. In this
instance, that would have meant peforming eight pairwise tests to
compare each bacterial infection to the control.  These results are
always biased by incomplete power \citep{Ding2010,
  Flutre2013}. Because hypothesis testing uses an arbitrary p-value
threshold to determine statistical significance, a gene with a p-value
below this threshold for one comparison but a p-value slightly above
this threshold for a separate comparison will be classified as
specific to the first when in reality the gene is behaving similarly
in both. As the number of pairwise comparisons increases, the problem
of incomplete power is exacerbated, i.e. a gene is more likely to be
statistically significant for some susbset of comparisons. This
increase in comparisons also decreases the ability to interpret the
results. A 3-way Venn diagram (and perhaps a 4- or 5-way) can be
interpreted, but this approach breaks down with additional
comparisons.

Another approach would be to directly compare the effect of infection
between two different groups of bacteria, e.g. compare the mean effect
of infection with mycobacteria versus the mean effect of infection
with non-mycobacteria (or virulent versus non-virulent bacteria). The
advantage of this approach is that it explicitly models the comparison
and returns a p-value, unlike the Venn diagram overlap
approach. However, there are two main downsides. First, statistical
significance can be driven by outliers. For example, in my study most
of the significantly differentially expressed genes between
mycobacteria and non-mycobacteria were actually genes which were
simply differentially expressed in response to infection with
\emph{S. typhimurium} and \emph{S. epidermidis}. Second, this limits
the potential results to the \emph{a priori} ideas of the analyst and
are not driven by the patterns in the actual data.

On the other end of the spectrum, a very data-driven approach would be
to use a clustering method such as hierarchical or k-means clustering
\citep{Eisen1998, Michaels1998}. These multivariate methods are able
to find the patterns of gene expression in the data, both expected and
unexpected; however, since they are not accompanied by any formal
hypothesis test, it is difficult to interpret which clusters of
co-expressed genes are the most interesting to report.

Since none of the standard genomics approaches were adequate for
properly comparing 8 bacterial infections, I instead used a joint
Bayesian model, implemented in the software package Cormotif, to
analyze the data \citep{Wei2015}. Conceptually, Cormotif combines the
clustering and pairwise testing approaches described above. Just like
the pairwise testing approach, the input to Cormotif are the pairwise
comparisons between each bacterial infection and the control
condition. However, to account for incomplete power, Cormotif models
the gene expression levels across all the pairwise comparisons to
identify the main gene expression patterns, conceptually similar to a
clustering analysis.

The Cormotif results for my study were informative. Most of the genes
were either differentially expressed or not after infection with any
of the bacteria (Fig. \ref{fig:joint-all}). The two most interesting
patterns in regards to understanding the innate immune response to MTB
were ``MTB'' and ``Virulent''
(Fig. \ref{fig:joint-18h},\ref{fig:joint-48h}).  The ``MTB'' pattern
included those genes which had a high posterior probability of being
differentially expressed in response to infection with MTB or closely
related species and a medium posterior probability of being
differentially expressed in response to ifnection with
\emph{M. smegmatis}, the nonvirulent mycobacteria. The ``Virulent''
pattern included genes which had a high posterior probability of being
differentially expressed in response to infection with any of the
bacteria except heat-inactivated MTB or BCG.

In terms of better understanding TB susceptibility, the main takeaway
from this study was the identification of hundreds of genes which are
differentially expressed in response specifically to infection with
MTB and related species but not other virulent bacteria. These genes
are candidates for containing genetic variants which affect TB
susceptibility. Furthermore, these genes could be targets for future
functional studies of how the innate immune system fights MTB and also
could give context to future results from genetic and functional
genomics studies of MTB infection. More generally, our methods are
informative to all future functional genomics studies. We were only
able to confidently isolate the effects of MTB infection by including
multiple other bacterial infections as comparison. Had we only
infected the macrophages with MTB and heat-inactivated MTB, we would
have made multiple misclassifications. We would have assigned
differences between the two infections as specific to a live, virulent
MTB; however, these gene expression changes were also induced by other
live bacteria. Similarly, we would never have known that a subset of
the genes which were differentially expressed in response to both MTB
and heat-inactivated MTB were actually specific to mycobacteria in
general. Not only was it important to include multiple bacterial
infections, but it was also critical to properly analyze the
results. Because the innate immune system is largely a general
response to infection, we expected most of the induced gene expression
changes to be similar across bacteria \citep{Huang2001, Boldrick2002,
  Nau2002, Jenner2005}.  Had we performed the straight-forward
approach of overlapping lists of differentially expressed genes from
comparing the individual infections to their controls, we would have
had identified lots of spurious differences in the innate immune
response caused by incomplete power. In contrast, by jointly modeling
the data with Cormotif \citep{Wei2015}, we were able to identify the
shared gene expression patterns in response to related bacteria. In
support of the generality of this approach, the Cormotif approach was
successfully applied to distinguish the effects of vitamin D and
bacterial lipopolysaccharide on the innate immune response between
individuals of African-American and European-American ancestry (note:
I was a co-author of the study) \citep{Kariuki2016}.

It should be noted that this method also has its caveats. First, its
strength of sharing information across the pairwise comparisons can
also be a negative because it will not identify genes with unique
expression patterns (Fig. \ref{fig:dusp14}). While useful for projects
with the aim of broadly characterizing the genome-wide gene expression
patterns for a given phenomenon, it is not well-suited for identifying
outlier genes. Second, because the algorithm is not deterministic,
Cormotif must be run multiple times to obtain the model with the
highest log likelihood. Because of this added complexity, using
Cormotif is more difficult to implement than more standard
differential expression approaches.

\section{Initial success classifying individuals susceptible to tuberculsosis and future directions}

In Chapter \ref{ch:tb-suscept}, I described my work investigating the
role of gene regulation in the innate immune system on TB
susceptibility (not yet published). Specifically, in order to
investigate how the innate immune cells of susceptible individuals
function compared to those of resistant individuals, we collected
primary dendritic cells (DCs) from individuals that had recovered from
TB (i.e. susceptible) and individuals that tested positive for latent
TB infections but had not developed TB (i.e. resistant). We infected
the DCs with MTB and performed RNA sequencing (RNA-seq) on the
infected and non-infected cells.

There were three main conclusions from this work. First, the
differences in gene expression levels between resistant and
susceptible individuals were primarily present in the non-infected
state and not 18 hours post-infection with MTB
(Fig. \ref{fig:limma}). This suggests that these gene expression
differences primarily affect the very early response to MTB
infection. Second, we discovered that the effect sizes measured in our
\emph{in vitro} experiment, whether comparing between resistant and
susceptible individuals or between the infected and non-infected
states, were negatively correlated with lower p-values from two genome
wide association studies (GWAS) of TB susceptibility \citep{Thye2010}
(Fig. \ref{fig:gwas}). This suggests that our \emph{in vitro} system
is a useful model for investigating the genetic basis of TB
susceptibility. Third, we trained a classifier based on the gene
expression levels in the non-infected state
(Fig. \ref{fig:classifier}). Using the threshold required to obtain a
100\% sensitivity (zero false negatives) in the training data, we
found that 11\% of healthy individuals from an independent study
\citep{Barreiro2012} were predicted to be susceptible to TB, very
close to the estimated population average of 10\% \citep{North2004,
  OGarra2013}. This suggests that isolating innate immune cells and
performing gene expression profiling could be a feasible test for TB
susceptibility.  The most obvious extension of this work is to conduct
a larger study with more susceptible individuals. Our current results
are only a proof-of-principle. With a larger study, we could properly
split the data into training and test sets to assure that the model is
not overfitting the data. On the one hand, since we identified that
the gene expression differences are only present in the non-infected
state and that these are sufficient for the performance of the
classifier, this future study would be simplified by not having to
perform the MTB infections. On the other hand, collecting a large
number of patient samples is always difficult, and it is even worse
when the individuals are currently healthy and thus not regularly
visiting the doctor like those recovered from a past case of active
TB. Hopefully these initial successful results will provide the
impetus for larger scale sample collection.

Another fruitful direction for future experiments would be to further
investigate the role of \emph{CCL1} in the innate immune response to
MTB and its role in TB susceptibility. While studies of this gene have
had mixed results \citep{Thuong2008, Tang2011, Ozdemir2013}, all the
studies, including my own \citep{Blischak2015}, have had small sample
sizes. In the first study of \emph{in vitro} differences in gene
expression between susceptible and resistant individuals, \emph{CCL1}
was found to be differentially expressed based on susceptibility
status \citep{Thuong2008}. Furthermore, the same study found that SNPs
nearby \emph{CCL1}were associated with TB susceptibility in an
independent cohort. In Chapter \ref{ch:tb}, I found that \emph{CCL1}
was one of the genes which changed expression level specifically in
response to infection with mycobacteria \citep{Blischak2015}. In
Chapter \ref{ch:tb-suscept}, I found that \emph{CCL1} was one of two
genes which had an effect size greater than 2 between susceptible and
resistant individuals in the non-infected state and also a p-value
less than 0.01 in two GWAS of TB susceptibility. There were many
differences between these studies (e.g. cell type, ethnicity of
donors, timepoints RNA was collected post-infection), yet \emph{CCL1}
was still a top hit in all three analyses. I believe this warrants
further investigation. As an example, one could use CRISPR/Cas
\citep{Du2016} to modify individual SNPs in THP-1 cells (a common cell
line model of monocytes) and test for differences in the response to
MTB infection. Another idea, since \emph{CCL1} is a secreted chemokine
\citep{Miller1992}, would be to add varying amounts of exogenous
\emph{CCL1} to the \emph{in vitro} system to detect an effect on the
innate immune response.

\section{Incorporating lessons from single cell pilot study for future studies of the genetic basis of gene expression noise and the response to bacterial infection}

In Chapter \ref{ch:singleCellSeq}, I described my work on single cell
RNA-seq (scRNA-seq) \citep{Tung2016}. scRNA-seq is a relatively new
technique \citep{Liang2014, Macaulay2014, Saliba2014, Grun2015,
  Stegle2015, Bacher2016} that enables the investigation of gene
regulatory changes at a much finer resolution than the bulk RNA-seq
projects I performed in Chapters \ref{ch:tb} and
\ref{ch:tb-suscept}. While this new technology is exciting, we must
exercise the same caution as when performing any large-scale genomics
experiment \citep{Auer2010, Leek2010, Gilad2015}. Early studies of the
Fluidigm C1 system for scRNA-seq that focused on the technical sources
of variation largely focused on the variation from well-to-well within
just one C1 chip \citep{Brennecke2013, Grun2014, Islam2014, Ding2015,
  Vallejos2015} ; whereas, the studies investigating biological
phenomena tended to use multiple C1 chips without addressing the
obvious confounding batch effects (this problem is nicely highlighted
by \citep{Hicks2015}). Before conducting large scale scRNA-seq
experiments, we first aimed to better understand the technical factors
affecting the design of such studies. To do so, we performed scRNA-seq
of three C1 chip replicates of three HapMap \citep{HapMap2005} Yoruba
individuals.

From these data, we learned many important lessons. First, by
performing subsampling analyses, we determined that sequencing
approximately 1.5 million reads for at least 75 cells from a given
individual was sufficient for detecting most expressed genes,
achieving a high correlation between the sum of the gene expression
levels across the single cells and the gene expression levels from
bulk sequencing of 10,000 cells, and achieving a high correlation
between the cell-to-cell variance in the gene expression levels across
the subset of single cells and the cell-to-cell variance measured in
all the single cells we collected for an individual (which ranged from
142 to 221) (Fig. \ref{fig:subsample}). Second, we observed technical
variation introduced from the processing of the C1 batches
(Fig. \ref{fig:batch}). While this was expected, we also observed
unexpected aspects of this batch effect. The ERCC spike-in controls
which were added to each well and could potentially be used to correct
for this effect across C1 chips was affected not only by technical
variation but also by the biological variation (differences between
individuals) (Fig. \ref{fig:ch04-s3}). This entanglement of the
technical and biological sources of variation renders the spike-ins
insufficient for modeling technical variation between C1 chips
(however they can still be used to model technical variation between
wells of the same C1 chip). This confounder occurred despite our use
of unique molecular identifiers (UMIs) to account for the bias
introduced by amplifying RNA from a small original source of just one
cell \citep{Kivioja2011, Islam2014}. In fact, we found that the
conversion of reads to unique molecules was affected by
inter-individual differences (Fig. \ref{fig:batch}). Third, even with
our small sample size of only three individuals, we were able to
identify inter-individual differences in the cell-to-cell gene
expression variance, or gene expression noise
(Fig. \ref{fig:variation}). This lends further support to the notion
that gene expression noise is a relevant factor that can affect
biological processes. Fourth, we demonstrated that we can use the
single nucleotide polymorphisms (SNPs) present in the RNA-seq reads to
identify the individual of origin for a given single cell
\citep{Jun2012} (Fig. \ref{fig:ch04-s8}). This enables us to use a
crossed-design where single cells from multiple individuals are
included on the same C1 chip and later each well is assigned to each
individual based on the RNA-seq reads obtained. Our initial nested
design was inefficient because we collected hundreds of single cells
per individual across the multiple technical replicates. From our
subsampling we knew that collecting 75 high quality single cells was
sufficient. With a crossed design, we can obtain about one C1 chip
worth of wells (96) while still modeling the technical variation
across C1 chips.

Given the promising results from our first study, our next study will
aim to further investigate the impact of genetic variation on gene
expression noise by measuring single cell gene expression levels in 60
individuals. The design of the study is informed by our previous
findings. First, we will put single cells from multiple individuals on
each C1 chip because we know we can identify the individual based on
the RNA-seq reads. Second, we will repeat each individual across C1
chips until they obtain on average 96 wells (e.g. one C1 chip) because
this will get us close to our target of 75 single cells after removing
low quality cells. Third, we will replace the ERCC spike-ins with RNA
from a distantly related model organism. With many more technical
spike-ins gene to measure, these will be more useful for modeling
technical variation \citep{Risso2014}. Using this study design, we'll
be able to efficiently measure gene expression noise from many
individuals while still properly accounting for technical variation.

Returning to the \emph{in vitro} models of bacterial infection from my
other studies, I can imagine future single cell studies that shed
further light on the innate immune response. While we infect the cells
at a multiplicity of infection of 1:1, some cells will still be
infected by multiple bacteria and others not infected at
all. Furthermore, there could be variation in this distribution of the
number of bacteria per cell across individuals. In order to
efficiently measure single cell gene expression in response to
infection, I would put uninfected cells from one individual on the
same C1 chip as the infected cells from a different individual. Also,
since the MTB H37Rv strain we typically use has a GFP tag, we could
use high-throughput fluorescence microscopy of each well to count the
number of bacteria per cell. With this high resolution data, we could
differentiate between inter-individual differences in the innate
immune response due to differences in the number of infected cells (or
the number of bacteria per cell) or differences in the innate immune
response in the infected cells.

\section{The importance of mitigating batch effects in any genomics experiment}

A common theme of all my projects is accounting for technical
biases. Although only Chapter \ref{ch:singleCellSeq} has a main focus
on mitigating batch effects, all my projects required close attention
to this problem. This is because all genomics studies need to account
for batch effects in both the design and analysis of the data,
otherwise the results are meaningless \citep{Auer2010, Leek2010,
  Gilad2015}. There will be signal in any large data set, but it will
only inform biological insight if the signal arises from the
biological processes being studied.

In Chapter \ref{ch:tb}, we collected a total of 156 RNA-seq
samples. During the batch processing, we ensured that the biological
factors of interest (bacterial infection, timepoint, individual) were
balanced to avoid introducing spurious signal. Furthermore, upon data
exploration, we observed that the processing batch and the RNA quality
score (RIN) were correlated with the first principal component (PC)
(Fig. \ref{fig:pca}). After regressing these two variables, the first
PC was the effect of timepoint and the second PC was the effect of
infection. Importantly, we obtained similar results with or without
protecting the variables of interest in the linear model when
regressing out the technical variables. This was a result of the
careful planning of the batch processing to avoid confounding
biological and technical variables.

In Chapter \ref{ch:tb-suscept}, I once again designed the batch
processing to balance the biological factors of interest
(susceptibility status, treatment, individual)
(Fig. \ref{fig:process}). Conveniently, this project did not have
large scale batch effects (Fig. \ref{fig:batch-effect}), likely due to
the smaller overall sample size of 48. However, accounting for a batch
effect was critical for training a classifier on the current data set
and testing it on an independent data set \citep{Barreiro2012}.  Not
only were the studies performed years apart, but the gene expression
levels were measured using different technologies. Thus I was only
able to compare the two studies after normalizing each sample
(Fig. \ref{fig:combined-dist}) and removing the large batch effect by
regressing the first PC of the combined data set
(Fig. \ref{fig:combined-pca}). Testing the classifier without
accounting for the batch effect would have given poor results simply
due to technical reasons.

In Chapter \ref{ch:singleCellSeq}, one of the main motivations for the
study was understanding the magnitude of the batch effect of
collecting scRNA-seq on separate C1 chips. While the technical effect
of C1 batch was smaller in magnitude compared to the biological effect
of individual (assessed using variance components analysis)
(Fig. \ref{fig:ch04-s3}), not including technical replicates would
attribute the substantial technical effect to the biological
effect. Just as we require replication for established genomic
protocols, it is also necessary to replicate scRNA-seq experiments,
especially since the standard ERCC spike-ins appear to be affected by
both biological and technical factors. Fortunately, we were able to
devise a strategy to reduce the required number of C1 replicates by
combining single cells from multiple individuals onto each C1 chip and
then replicating the multiple individuals across multiple chips
(Fig. \ref{fig:ch04-s8}). This crossed design accounts for batch
effects while minimizing the required replication.

In summary, technical batch effects need to be considered from the
initial design of a genomics experiments through to the data analysis
and interpretation of the results.

\section{Concluding remarks}

First, I have identified hundreds of genes specifically involved in
fighting MTB infections. More broadly, I have demonstrated that a
joint Bayesian model is an effective tool for analyzing the results of
genomic studies with many conditions. Second, I have demonstrated that
the gene expression levels in non-infected DCs may be able to predict
susceptiblity to TB. Third, I have determined an effective study
design for future single cell studies that accounts for technical
batch effects while simultaneously decreasing the necessary sample
size. Overall my results are informative not only for understanding
how differences in the innate immune response confer susceptibility or
resistance to TB, but also inform the design and analysis of any
functional genomics experiment.
