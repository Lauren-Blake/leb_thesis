\chapter{Conclusion}\label{conclusion}

A major promise of the field of human genetics is to better understand human complex traits and diseases. As described in previous chapters, a properly designed study can facilitate this biological understanding. In Chapter 2, the use of matched tissues between species allowed us to estimate the contribution of DNA methylation to conserved inter-tissue differences in gene expression (REF).  We also found that conserved gene regulatory differences are more likely to be involved in gene regulatory networks and protein-protein networks than non-conserved differences. In Chapter 3, we leveraged matched induced Pluripotent Stem Cell (iPSC) panels from humans and chimpanzees (REF). In this developmental timecourse from iPSCs to the endoderm stage, my collaborators and I found high conservation during the differentiation. Furthermore, we reported a reduction in regulatory variation at the primitive streak stage in a significant number of genes. Due to our study design and careful analysis of technical variables, we were able to provide evidence that this canalization is driven by biological, rather than technical, factors. Finally, in Chapter 4, I discussed power to identify biologically-based predictors in the context of a longitudinal study design. 
