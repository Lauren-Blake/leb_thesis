\chapter{Introduction}

The field of Human Genetics aims to discover the genetic basis of the
variation observed in human phenotypes. The difficulty of this goal
depends on the genetic architecture of the phenotype. On the one hand
are monogenic traits (also referred to as Mendelian traits), which are
caused by mutations in a single gene. Isolating the causal gene is
relatively tractable. On the other hand are polygenic (or complex)
traits, which are the result of many genes acting in
concert. Furthermore, the genes can interact with each other in
non-additive manner (gene-gene interactions, or epistasis) and the
environment can play a significant role (gene-environment
interactions).

As an example, consider human height. A disabling mutation in just one
gene, the growth hormone receptor (HGR), nullifies the effect of
growth hormone leading to very short stature and other metabolic
abnormalities (Laron Syndrome). Because of its easy to identify
phenotype and single gene origin, the genetic basis of Laron Syndrome
was discovered in the late 1980s using a small number of pedigrees. In
contrast, the considerable variability in height in the human
population is not caused by rare mutations in a single or few genes,
but instead is due to the aggregate effect of many mutations of small
effect size interacting with each other and the environment
(e.g. diet, pollution, etc.). Thus although height has been determined
to be highly heritable, genetic studies involving hundreds of
thousands of individuals that identified thousands of associated
variants which affect height still only explain a small percentage of
the heritable variation. Larger studies with increased power will only
continue to find associated variants with even smaller effect size (or
otherwise they would have already been discovered), thus the genetic
basis of a highly polygenic trait subject to millennia of natural
selection may unsatisfyingly be that most of the genome makes a minute
contribution to the height of an individual.

The current state-of-the-art technique of mapping genetic variants
that affect a polygenic trait is the genome-wide association study
(GWAS). This technique was made possible by the sequencing of the
human genome and the cataloging of the common genetic variation
segregating in the human population (the latter done via the
International HapMap Project). For a GWAS, individuals are phenotyped
(e.g. height is measured) and genotyped at millions of common single
nucleotide polymorphisms (SNPs).  Then each SNP is tested individually
for an association with the trait measurements via a linear regression
or related statistical technique. Similarly, for a binary trait such
as cases with a disease versus controls without a disease, the
phenotype is the presence or absence of a disease and each SNP is
tested for association with a logistic regression or related
statistical technique. GWAS have identified many genetic variants
affecting a diverse set of human polygenic traits, especially as the
sample sizes for GWAS increased into the hundreds of
thousands. Nevertheless, their results have several limitations.

As mentioned above, one of the main issues with GWAS results is the
small effect size of the associated SNPs on the trait of interest. The
hope of finding these SNPs is that they will be useful for predicting
the trait (e.g. how likely are you to develop diabetes). However, with
such small effect sizes, they have little predictive power and thus
are generally not clinically actionable. These disappointing results
could be due to limitations in our knowledge when designing the study
and modeling the data. For example, when recruiting study
participants, it is impossible to record every possible environmental
factor that could have contributed to each person’s trait
value. Furthermore, in case-control studies, the controls will likely
include a subset of individuals that have yet to develop the
disease. Similarly, when modeling the genetic associations, most
models assume an isolated additive effect of each variant on the
trait. This simplifying assumption is made such that the statistical
test is tractable and interpretable. However, it is missing the
contribution of any gene-gene or gene-environment interactions. On the
other hand, the disappointing results of GWAS may not be due to
limitations of the approach, but simply reflect the actual biology of
polygenic traits. Mutations with strong effect, such as those that
disable the GHR and cause Laron Syndrome, are often disruptive to the
complex network of biochemical reactions that sustain a living
individual. For this reason, they face strong negative selection and
are often rare in the population. In contrast, mutations with small
effect on a trait are more likely to be neutral or slight favorable,
and thus are able to rise to higher allele frequencies in the
population. Over millions of years of evolution, the many variants of
small effect could give rise to the large variation in phenotypes
observed today, e.g. the difference in height between a 5 foot person
and a 7 foot person. Supporting this view, when all SNPs assayed in an
experiment are used to explain heritability, known as the “chip”
heritability, this estimate is close to the observed
heritability. This suggests that highly polygenic traits like human
height are indeed the result of millions of variants of small effect
size.

Beyond the ability to predict a disease outcome or trait value,
another goal of GWAS is to elucidate the underlying biological
mechanisms which ultimately determine the trait. This has proven
difficult because most GWAS hits do not affect the protein-coding
sequence of a gene, for which it would be straightforward to predict
and test the effect this would have on gene function, but instead the
associated SNPs are located in non-coding regions of the genome. It is
much more difficult to predict the effect of these variants because
there is no simple code to translate changes in non-coding
sequence. This has motivated the study of gene regulation in the field
of Human Genetics.

Gene regulation refers to how cells control which genes are turned on
and to what extent. This is critical because all cells in the human
body contain the same genomic material (ignoring the complications of
somatic recombination in certain immune cells and somatic mutations in
general). Thus in order for a liver cell to function differently than
a skin cell, the two cells must have different gene expression
levels. These gene regulatory differences are established during
development as an organism grows from an initial single
cell. Signaling molecules, initially from the mother but subsequently
produced by the offspring’s cells, bind to the receptors of a cell to
initiate signal transduction cascades that ultimately lead to
activation of a transcription factor which binds to DNA at its
degenerate binding sites across the genome to modulate the expression
of many genes. As development continues and cells differentiate into
their final tissue type, the gene expression levels are maintained by
the gene regulatory network established by the transcription factors
active in that cell type.

Just as differences in gene regulation generates extreme diversity in
cellular function among cells with identical genomes in a given
organism, a long standing hypothesis is that differences among humans
and the differences between humans and our closest evolutionary
relatives, the great apes, are due to mutations that affect not the
protein-coding sequence but instead mutations which affect the
spatiotemporal expression of genes. This theory was originally
proposed because of the high similarity of protein-coding sequences
between humans and chimpanzees, and is supported by the finding of
mainly non-coding SNPs from GWAS.

Understanding which transcription factors establish and maintain a
given cellular identity is quite difficult. However, even without this
knowledge, it is possible to learn about the regulatory state of a
given cell type. First, it is possible to measure genome-wide gene
expression levels using technologies like microarrays or RNA
sequencing (more on that below). Second, it is possible to interrogate
the non-coding regions of the genome by measuring chromatin
marks. Chromatin marks are deposited by chromatin-remodeling enzymes
which are recruited by the transcription factors active in the
cell. The most common are methylation of the cysteine base in CpG
dinucleotides (DNA methylation) or chemical modification of the tails
of the protein octamers (histones) which DNA is wrapped around. These
marks signal the state of the region, e.g. active or repressed, and
help to maintain the current state. Histone marks can be assayed with
chromatin immunoprecipitation followed by sequencing (ChIP-seq), and
DNA methylation can be assayed with specialized microarrays or
bisulphite sequencing. Using these technologies, it is possible to
learn about the function of the non-coding SNPs discovered by GWAS.

As an aside, it should be noted that there is a lot of confusion about
the role of chromatin marks and their effect on gene
expression. Chromatin marks are not causal. Instead, they are signs of
a given chromatin state, and at best help maintain that state. As an
analogy, consider viewing a stretch of highway from a helicopter. If
you observe orange signs and barrels, you can conclude that this
section of the highway is a construction zone. Furthermore, because
they notify the motorists to slow down and to merge into one lane, you
can conclude that the constructions signs and barrels help this
section to maintain the characteristics of a construction
zone. However, you would not conclude that the signs and barrels
caused this section of highway to be a construction zone. The decision
to work on this section of road was made by local government officials
and contractors after observing the conditions of the road and
receiving complaints from citizens. In gene regulation, the chromatin
marks are the constructions signs and barrels. If you observe
activating chromatin marks, you can conclude that the nearby gene is
expressed and that the chromatin marks are helping maintain this
transcriptional activity. However, it is the result of transcription
factors receiving input from outside the cell that caused these active
chromatin marks to be established and the gene to be expressed.

Thus using these chromatin marks enables the deciphering of the
non-coding regions of the genome. While not as easy initially
envisioning a readable “histone code,” much progress has been
made. The ENCODE Project assayed the gene expression and many
chromatin marks in a large variety of cell types. Using a hidden
Markov model (HMM), they were able to define distinct regions of the
genome in each of the cell types they collected. This now provides the
context-specificity required to predict and test the effect of
non-coding SNPs identified in GWAS. For example, a GWAS hit for type
II diabetes could be potentially affecting gene expression in the
liver, adipose tissue, brain, or beta cells of the pancreas. If
chromatin profiling reveals that SNP is located in an enhancer region
in only one of those tissues, that would inform the follow-up
experiments to perform. Encouragingly, this sort of relationship is
observed generally. That is, GWAS hits for given disease are more
likely to be found in gene regulatory regions of the genome specific
to tissues relevant to the disease pathogenesis. Furthermore,
knowledge of these genomic annotations has been successfully used as
prior information to increase the power to detect associations in a
GWAS \citep{Wang2016}.

While knowing that an associated SNP is located in an enhancer region
in a particular cell type is extremely helpful for generating testable
hypotheses, it still leaves many unanswered questions. While it is
usually assumed that a variant is affecting the most nearby gene,
there is no guarantee this is true. And even if that assumption is
true, it is unknown which allele is associated with higher
expression. A direct method for addressing these uncertainties is
expression quantitative trait loci (eQTL) mapping (note that the name
is a misnomer; early eQTL studies were performed using linkage in
pedigrees, but current eQTL studies are tests of association in
unrelated individuals like a typical GWAS). In this approach the
phenotype of interest is the expression level of a gene. To reduce the
multiple testing burden (and also because regulatory variants are
often closer to the gene they affect), most eQTL studies test for
eQTLs nearby the transcription start site of each gene. To date, eQTL
studies have been performed in many cell types. Reassuringly, eQTLs
are more likely to be GWAS associated SNPs, consistent with the idea
that GWAS hits in non-coding regions are affecting gene
expression. Furthermore, by combining eQTL results from many tissues
collected by the GTEx Consortium with GWAS results, it is possible to
determine the tissue(s) most affecting a given disease by finding
which tissue is enriched for tissue-specific eQTLs that are also GWAS
hits for the disease \citep{Ongen2016}.

A common functional genomics technique is RNA sequencing (RNA-seq). It
is an efficient method for interrogating cellular function by
measuring genome-wide gene expression levels. RNA-seq has multiple
advantages over its predecessor, gene expression microarrays. For
example, it is not as limited by genome annotations and has a higher
dynamic range. Most importantly for Human Genetics applications, any
polymorphisms present in the coding regions in a population being
studied will be present in the RNA-seq reads. These can be used to
verify the identity of the individual being sequenced (i.e. avoid
sample swaps and contamination), and also to increase power in eQTL
studies by comparing the allele-specific expression measurements to
the eQTL effects.
