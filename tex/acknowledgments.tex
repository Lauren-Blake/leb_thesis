\acknowledgments

I am truly grateful for all the assistance and guidance I have
received during my PhD studies. None of my projects would have been
feasible without the help of my collaborators.

I am very thankful to my advisor, Yoav Gilad. Not only has he had the
largest influence on the development of my scientific thinking, but
just as important, he made my graduate school experience enjoyable. He
found the right balance of letting me explore ideas but also reminding
me to stay on track. I could not have asked for a better PhD advisor.

I am thankful to my committee members. They have also been
instrumental in my path to graduation. My chair, Joe Thornton,
consistently reminded me to remember the ``big picture'' when
describing my research, a very important lesson since I have the
tendency to focus on the details. Matthew Stephens has always had
great patience when explaining statistical concepts to me, from the
basic to the advanced. John Novembre was always ready with a great
piece of advice to improve my projects (e.g. the figures in Chapter 2
are much more interpretable after incorporating his suggestions).

I really enjoyed being a member of the Gilad lab. I learned so much
from my labmates, past and present, and we also had a lot of fun over
the years. I have to specifically thank Darren Cusanovich and Irene
Gallego Romero, who invested a lot of time training me how to think
like a scientist when I was a junior graduate student.

Even more broadly, I have greatly benefited from the Human Genetics
community in Cummings Life Science Center. From the formal
collaborations of TAing with Mark Abney and doing a project with
Silvia Kariuki and Anna Di Rienzo, to the informal interactions with
other professors, postdocs, and graduate students. I am so grateful to
have had the opportunity to spend my PhD years in such a collegial
environment.

My projects on tuberculosis would not have been possible without the
help of Ludovic Tailleux and Luis Barriero. I am thankful to Ludo for
expertly performing the many bacterial infections required for our
studies, and for all his work handling IRB applications and patient
recruitments. I am thankful to Luis for his insightful advice and
enthusiasm.

The single cell sequencing project was also a group effort. I am
thankful for PoYuan Tung for expertly performing all the experimental
work and her extensive knowledge of the single cell literature, for
Joyce Hsiao for her ability to tackle tough statistical problems, and
to Dave Knowles and Jonathan Pritchard for their insightful advice.

I was very fortunate to have made some great friends during my time at
University of Chicago. I wasn't expecting to have so much fun during
my PhD. I am going to really miss them as we all graduate and move on
to other things.

My family has been super supportive of me my whole life. I am
especially appreciative of my parents. They've supported me even
during those times I was not enjoying graduate school.

Lastly, I cannot thank my wife enough. She has made so many sacrifices
so that I could complete my research and earn my PhD. None of this
would have been possible without her.
