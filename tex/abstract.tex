\abstract
A primary aim of the human genetics is to determine how genetic variation impacts phenotypic variation, including in complex traits and diseases. Understanding this relationship will ultimately allow the field understand the molecular basis of complex traits and better diagnose and treat human diseases. To dissect this relationship, human geneticists have leveraged comparisons between humans and other primates, as well as between different groups of humans. To maximize the utility of these functional genomics studies, proper study design must be deployed. Indeed, the primate comparative studies in Chapters 2 and 3 highlight study design challenges and potential solutions. Furthermore, this work demonstrates that adherence to key study design principles helps elucidate biological insight. In Chapter 4, I apply these solutions to a new problem, distinguishing individuals with eating disorders at high risk of rehospitalization from those with lower risk. In the final chapter, I discuss lessons learned and next steps for using functional genomics to study eating disorders.
