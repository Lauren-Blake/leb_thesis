\abstract
A primary aim of human genetics is to determine how genetic variation impacts phenotypic variation, including in complex traits and diseases. Understanding this relationship will ultimately allow us to understand the molecular basis of complex traits and better diagnose and treat human diseases. To dissect this relationship, human geneticists have leveraged comparisons between humans and other primates, as well as between different groups of humans. To maximize the utility of functional genomics studies within and between species, proper study design is essential. In Chapters 2 and 3, I describe two primate comparative studies that highlight a variety of study design challenges and discuss potential solutions.These studies demonstrate that adherence to key study design principles can facilitate biological insight. In Chapter 4, I apply these principles to a new problem, distinguishing individuals with eating disorders at high risk of rehospitalization from those with lower risk. In the final chapter, I discuss lessons learned and suggest the next steps for using functional genomics to study eating disorders.
