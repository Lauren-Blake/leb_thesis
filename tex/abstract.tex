\abstract
A major goal of human genetics is to characterize the role of genetic
variation on complex, polygenic phenotypes. With the discovery from
genome-wide association studies (GWAS) that many associated variants
have a small effect size and are located in non-coding regions of the
genome, there has been a large effort to collect functional genomics
data. The hope is that a better understanding of how the genome
functions in diverse developmental states and environments will
provide insight into the context-specific activity of associated
non-coding variants. My research applies this paradigm to the complex
phenotype of susceptibility to develop tuberculosis (TB). It has been
estimated that 10\% of individuals infected with \emph{Mycobacterium
tuberculosis} (MTB) progress to active disease. Despite being
heritable, very few genetic variants have been associated with
susceptibility to TB. For my studies, I use RNA sequencing (RNA-seq)
to interrogate genome-wide transcript levels in \emph{in vitro}
cellular models. In Chapter \ref{ch:tb}, I use a joint Bayesian model
to idenitfy genes which are differentially expressed in macrophages
only after infection with MTB and related mycobacteria, but not other
bacterial pathogens. In Chapter \ref{ch:tb-suscept}, I build a support
vector machine model to classify individuals as susceptible or
resistant to TB based on the gene expression levels in their dendritic
cells. In Chapter \ref{ch:singleCellSeq}, I characterize the technical
variation introduced by batch processing of single cell RNA-seq
(scRNA-seq) and propose an effective study design that accounts for
technical variation while minimizing replication.  In addition to
providing insight into the genes important for the innate immune
response to MTB infection, my work is informative for the design and
analysis of future functional genomics experiments.
